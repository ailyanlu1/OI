\documentclass[aspectratio=43]{beamer}
\usepackage{ctex}
\usepackage[utf8]{inputenc}
\usepackage[T1]{fontenc}

\title{k步最短路}
\date{\today}
\author[Alice]{kririae}

\usetheme{material}

\useDarkTheme
\usePrimary{38E4AE}{311B92}{FFFFFF}
\useAccent{C6FF00}{000000}

\begin{document}

\begin{frame}
\titlepage
\end{frame}

\section{题目说明}
\begin{frame}{题目说明}
\begin{card}
一个带权有向图,求a到b经过k个点的最短路
\newline
\newline
n\leq 300, k\leq 10^{18}
\end{card}
\end{frame}

\section{题解说明}
\begin{frame}{分析}
\begin{card}
一看这个数据,清楚的人就知道了 \newline
摆明了是让你用floyd
\newline
既然今天是学习倍增...就从倍增的角度来思考这道题
\newline
倍增本身并不能和最短路扯上什么关系,
那就从最短路优化的角度来看
首先给出结论吧
倍增之前的时间复杂度是O(k\cdot N^3)
优化之后是O(\log{k}\cdot N^3)
\end{card}
\end{frame}

\begin{frame}{分析}
\begin{card}
floyd的状态转移方程式f[i][j] = \mathrm{min}(f[i][k], f[k][j])
\newline
网上有一篇比较好的题解\newline\url{http://blog.csdn.net/jarily/article/details/12089629}

\end{card}
\end{frame}

\begin{frame}{SPFA}
可能有错
\begin{card}
既然SPFA是动态规划,经典的“背包问题选取k个物品的最大值”,那这里修改一下SPFA的状态,修改为dist[k][i],表示经过了k个点到达了点i的最大值,即经过k步。\newline
转移方程也修改一下
\newline
f[k][t_i] = \mathrm{max}(f[k - 1][f_i] + c_i, f[k][t_i])
, f[k][i] = 0, O(\frac{k^2}{n})
\end{card}
\end{frame}
\end{document}
