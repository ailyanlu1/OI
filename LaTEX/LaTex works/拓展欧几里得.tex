\documentclass[aspectratio=43]{beamer}
\usepackage{ctex}
\usepackage[utf8]{inputenc}
\usepackage[T1]{fontenc}
\usepackage{listings}

\title{欧几里得算法}
\date{\today}
\author[Alice]{成都石室中学 \qquad kririae}

\usetheme{material}

\useLightTheme
\usePrimary{B60D34}{50C5B7}{FFFFFF}
\useAccent{C6FF00}{000000}

\begin{document}

\begin{frame}
\titlepage
\end{frame}

\section{基本}
\begin{frame}{预备知识1}
\begin{card}
欧几里得算法: \newline
$gcd(a, b) = gcd(b, a \;mod\; b)$ \newline
其实就是小学时候学习的辗转相除法的递归写法
\end{card}
\end{frame}

\section{基本}
\begin{frame}{预备知识2}
\begin{card}
mod的定义: \newline
$a (mod\,b) = a - \lfloor\frac{a}{b}\rfloor b \newline
\Leftrightarrow a = a \;mod\; b + \lfloor\frac{a}{b}\rfloor b$
\end{card}
\end{frame}

\section{拓展欧几里得}
\begin{frame}{拓展欧几里得}
\begin{card}
求解不定方程
$ax + by = gcd(a, b)$
可以求出递推公式: \newline
ax_1 + by_1 = gcd(a, b) \newline
    \Rightarrow bx_2 + (a\;mod\;b)y_2 = gcd(b, a\;mod\;b) \newline
\Rightarrow ax_1 + by_1 = bx_2 + (a\;mod\;b)y_2 \newline
a \;mod\; b = a - \lfloor\frac{a}{b} \rfloor b\newline
\Rightarrow ax_1 + by_1 = bx_2 + (a - \lfloor\frac{a}{b}\rfloor b)y_2 \newline
\Rightarrow ax_1 + by_1 = ay_2 + b(x_2 - \lfloor\frac{a}{b}\rfloor y_2) \newline
x_1 = y_2 \newline
y_1 = x_2 - \lfloor\frac{a}{b}\rfloor y_2
\end{card}
\end{frame}

\section{基本}
\begin{frame}{小小的总结}
\begin{card}
$ax-kc=b$,先假设$ax-kc=gcd(a, c)$,现在的解是乘$\frac{gcd(a, c)}{b}$之后的,在输出结果的时候需要保证是正数。
\end{card}
\end{frame}

\section{基本}
\begin{frame}{小小的总结2}
\begin{card}
需要证明$ax \equiv b \; (mod \; c)$
支持换元
$$\Rightarrow ax - b \equiv 0 \; (mod \; c)$$
由此可得
$$\Rightarrow ax - b = ck$$
移项可得拓展欧几里得式
$$ ax-  ck = b$$
假设
$$ ax - ck = gcd(a, c)$$
\end{card}
\end{frame}

\section{小小的总结2}
\begin{frame}{小小的总结2}
\begin{card}
然后$exgcd(a, c, x, -k)$求出x和-k。
左右同时乘$\frac{b}{gcd(a, c)}$,解出x的值。
\end{card}
\end{frame}

\section{小小的总结2}
\begin{frame}{小小的总结2}
\begin{card}
如何证明,当且仅当$gcd(a, c) | b$时,才存在解?
\end{card}
\end{frame}

\section{小小的总结2}
\begin{frame}{小小的总结2}
\begin{card}
已经看到了,上面的公式中,结果需要乘一个数$\frac{b}{gcd(a, b)}$
当且仅当这个数字是整数,才行。
\end{card}
\end{frame}
\end{document}